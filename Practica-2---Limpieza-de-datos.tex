% Options for packages loaded elsewhere
\PassOptionsToPackage{unicode}{hyperref}
\PassOptionsToPackage{hyphens}{url}
%
\documentclass[
]{article}
\usepackage{amsmath,amssymb}
\usepackage{lmodern}
\usepackage{ifxetex,ifluatex}
\ifnum 0\ifxetex 1\fi\ifluatex 1\fi=0 % if pdftex
  \usepackage[T1]{fontenc}
  \usepackage[utf8]{inputenc}
  \usepackage{textcomp} % provide euro and other symbols
\else % if luatex or xetex
  \usepackage{unicode-math}
  \defaultfontfeatures{Scale=MatchLowercase}
  \defaultfontfeatures[\rmfamily]{Ligatures=TeX,Scale=1}
\fi
% Use upquote if available, for straight quotes in verbatim environments
\IfFileExists{upquote.sty}{\usepackage{upquote}}{}
\IfFileExists{microtype.sty}{% use microtype if available
  \usepackage[]{microtype}
  \UseMicrotypeSet[protrusion]{basicmath} % disable protrusion for tt fonts
}{}
\makeatletter
\@ifundefined{KOMAClassName}{% if non-KOMA class
  \IfFileExists{parskip.sty}{%
    \usepackage{parskip}
  }{% else
    \setlength{\parindent}{0pt}
    \setlength{\parskip}{6pt plus 2pt minus 1pt}}
}{% if KOMA class
  \KOMAoptions{parskip=half}}
\makeatother
\usepackage{xcolor}
\IfFileExists{xurl.sty}{\usepackage{xurl}}{} % add URL line breaks if available
\IfFileExists{bookmark.sty}{\usepackage{bookmark}}{\usepackage{hyperref}}
\hypersetup{
  pdftitle={Tipología y ciclo de vida de los datos: Practica 2},
  pdfauthor={Autor: Borja López Gómez y Sergio Beltrán Nuez},
  hidelinks,
  pdfcreator={LaTeX via pandoc}}
\urlstyle{same} % disable monospaced font for URLs
\usepackage[margin=1in]{geometry}
\usepackage{color}
\usepackage{fancyvrb}
\newcommand{\VerbBar}{|}
\newcommand{\VERB}{\Verb[commandchars=\\\{\}]}
\DefineVerbatimEnvironment{Highlighting}{Verbatim}{commandchars=\\\{\}}
% Add ',fontsize=\small' for more characters per line
\usepackage{framed}
\definecolor{shadecolor}{RGB}{48,48,48}
\newenvironment{Shaded}{\begin{snugshade}}{\end{snugshade}}
\newcommand{\AlertTok}[1]{\textcolor[rgb]{1.00,0.81,0.69}{#1}}
\newcommand{\AnnotationTok}[1]{\textcolor[rgb]{0.50,0.62,0.50}{\textbf{#1}}}
\newcommand{\AttributeTok}[1]{\textcolor[rgb]{0.80,0.80,0.80}{#1}}
\newcommand{\BaseNTok}[1]{\textcolor[rgb]{0.86,0.64,0.64}{#1}}
\newcommand{\BuiltInTok}[1]{\textcolor[rgb]{0.80,0.80,0.80}{#1}}
\newcommand{\CharTok}[1]{\textcolor[rgb]{0.86,0.64,0.64}{#1}}
\newcommand{\CommentTok}[1]{\textcolor[rgb]{0.50,0.62,0.50}{#1}}
\newcommand{\CommentVarTok}[1]{\textcolor[rgb]{0.50,0.62,0.50}{\textbf{#1}}}
\newcommand{\ConstantTok}[1]{\textcolor[rgb]{0.86,0.64,0.64}{\textbf{#1}}}
\newcommand{\ControlFlowTok}[1]{\textcolor[rgb]{0.94,0.87,0.69}{#1}}
\newcommand{\DataTypeTok}[1]{\textcolor[rgb]{0.87,0.87,0.75}{#1}}
\newcommand{\DecValTok}[1]{\textcolor[rgb]{0.86,0.86,0.80}{#1}}
\newcommand{\DocumentationTok}[1]{\textcolor[rgb]{0.50,0.62,0.50}{#1}}
\newcommand{\ErrorTok}[1]{\textcolor[rgb]{0.76,0.75,0.62}{#1}}
\newcommand{\ExtensionTok}[1]{\textcolor[rgb]{0.80,0.80,0.80}{#1}}
\newcommand{\FloatTok}[1]{\textcolor[rgb]{0.75,0.75,0.82}{#1}}
\newcommand{\FunctionTok}[1]{\textcolor[rgb]{0.94,0.94,0.56}{#1}}
\newcommand{\ImportTok}[1]{\textcolor[rgb]{0.80,0.80,0.80}{#1}}
\newcommand{\InformationTok}[1]{\textcolor[rgb]{0.50,0.62,0.50}{\textbf{#1}}}
\newcommand{\KeywordTok}[1]{\textcolor[rgb]{0.94,0.87,0.69}{#1}}
\newcommand{\NormalTok}[1]{\textcolor[rgb]{0.80,0.80,0.80}{#1}}
\newcommand{\OperatorTok}[1]{\textcolor[rgb]{0.94,0.94,0.82}{#1}}
\newcommand{\OtherTok}[1]{\textcolor[rgb]{0.94,0.94,0.56}{#1}}
\newcommand{\PreprocessorTok}[1]{\textcolor[rgb]{1.00,0.81,0.69}{\textbf{#1}}}
\newcommand{\RegionMarkerTok}[1]{\textcolor[rgb]{0.80,0.80,0.80}{#1}}
\newcommand{\SpecialCharTok}[1]{\textcolor[rgb]{0.86,0.64,0.64}{#1}}
\newcommand{\SpecialStringTok}[1]{\textcolor[rgb]{0.80,0.58,0.58}{#1}}
\newcommand{\StringTok}[1]{\textcolor[rgb]{0.80,0.58,0.58}{#1}}
\newcommand{\VariableTok}[1]{\textcolor[rgb]{0.80,0.80,0.80}{#1}}
\newcommand{\VerbatimStringTok}[1]{\textcolor[rgb]{0.80,0.58,0.58}{#1}}
\newcommand{\WarningTok}[1]{\textcolor[rgb]{0.50,0.62,0.50}{\textbf{#1}}}
\usepackage{graphicx}
\makeatletter
\def\maxwidth{\ifdim\Gin@nat@width>\linewidth\linewidth\else\Gin@nat@width\fi}
\def\maxheight{\ifdim\Gin@nat@height>\textheight\textheight\else\Gin@nat@height\fi}
\makeatother
% Scale images if necessary, so that they will not overflow the page
% margins by default, and it is still possible to overwrite the defaults
% using explicit options in \includegraphics[width, height, ...]{}
\setkeys{Gin}{width=\maxwidth,height=\maxheight,keepaspectratio}
% Set default figure placement to htbp
\makeatletter
\def\fps@figure{htbp}
\makeatother
\setlength{\emergencystretch}{3em} % prevent overfull lines
\providecommand{\tightlist}{%
  \setlength{\itemsep}{0pt}\setlength{\parskip}{0pt}}
\setcounter{secnumdepth}{-\maxdimen} % remove section numbering
\ifluatex
  \usepackage{selnolig}  % disable illegal ligatures
\fi

\title{Tipología y ciclo de vida de los datos: Practica 2}
\author{Autor: Borja López Gómez y Sergio Beltrán Nuez}
\date{Diciembre 2021}

\begin{document}
\maketitle

{
\setcounter{tocdepth}{2}
\tableofcontents
}
\newpage

\begin{center}\rule{0.5\linewidth}{0.5pt}\end{center}

\hypertarget{introducciuxf3n}{%
\section{Introducción}\label{introducciuxf3n}}

\begin{center}\rule{0.5\linewidth}{0.5pt}\end{center}

\hypertarget{presentaciuxf3n}{%
\subsection{Presentación}\label{presentaciuxf3n}}

En esta práctica se elabora un caso práctico orientado a aprender a
identificar los datos relevantes para un proyecto analítico y usar las
herramientas de integración, limpieza, validación y análisis de las
mismas.

\hypertarget{competencias}{%
\subsection{Competencias}\label{competencias}}

En esta práctica se desarrollan las siguientes competencias del Máster
de Data Science: * Capacidad de analizar un problema en el nivel de
abstracción adecuado a cada situación y aplicar las habilidades y
conocimientos adquiridos para abordarlo y resolverlo. * Capacidad para
aplicar las técnicas específicas de tratamiento de datos (integración,
transformación, limpieza y validación) para su posterior análisis

\hypertarget{objetivos}{%
\subsection{Objetivos}\label{objetivos}}

Los objetivos concretos de esta práctica son: * Aprender a aplicar los
conocimientos adquiridos y su capacidad de resolución de problemas en
entornos nuevos o poco conocidos dentro de contextos más amplios o
multidisciplinares.Tipología y ciclo de vida de los datos Práctica 2 pág
1

\begin{itemize}
\item
  Saber identificar los datos relevantes y los tratamientos necesarios
  (integración, limpieza y validación) para llevar a cabo un proyecto
  analítico.
\item
  Aprender a analizar los datos adecuadamente para abordar la
  información contenida en los datos.
\item
  Identificar la mejor representación de los resultados para aportar
  conclusiones sobre el problema planteado en el proceso analítico.
\item
  Actuar con los principios éticos y legales relacionados con la
  manipulación de datos en función del ámbito de aplicación.
\item
  Desarrollar las habilidades de aprendizaje que les permitan continuar
  estudiando de un modo que tendrá que ser en gran medida autodirigido o
  autónomo.
\item
  Desarrollar la capacidad de búsqueda, gestión y uso de información y
  recursos en el ámbito de la ciencia de datos
\end{itemize}

\hypertarget{descripciuxf3n-de-la-pec-a-realizar}{%
\subsection{Descripción de la PEC a
realizar}\label{descripciuxf3n-de-la-pec-a-realizar}}

La prueba está estructurada en 1 ejercicio teórico/práctico y 1
ejercicio práctico que pide que se desarrolle la fase de preparación en
un juego de datos.\\
Deben responderse todos los ejercicios para poder superar la PEC.

\hypertarget{recursos}{%
\subsection{Recursos}\label{recursos}}

Los siguientes recursos son de utilidad para la realización de la
práctica: * Calvo M., Subirats L., Pérez D. (2019). Introducción a la
limpieza y análisis de los datos. Editorial UOC.

\begin{itemize}
\item
  Megan Squire (2015). Clean Data. Packt Publishing Ltd.
\item
  Jiawei Han, Micheine Kamber, Jian Pei (2012). Data mining: concepts
  and techniques. Morgan Kaufmann.
\item
  Jason W. Osborne (2010). Data Cleaning Basics: Best Practices in
  Dealing with Extreme Scores. Newborn and Infant Nursing Reviews; 10
  (1): pp.~1527-3369.
\item
  Peter Dalgaard (2008). Introductory statistics with R. Springer
  Science \& Business Media.
\item
  Wes McKinney (2012). Python for Data Analysis. O'Reilley Media, Inc.
\item
  Tutorial de Github
  \url{https://guides.github.com/activities/hello-world}.
\end{itemize}

\newpage

\begin{center}\rule{0.5\linewidth}{0.5pt}\end{center}

\hypertarget{descripciuxf3n-del-dataset}{%
\section{1.Descripción del dataset}\label{descripciuxf3n-del-dataset}}

\begin{center}\rule{0.5\linewidth}{0.5pt}\end{center}

\#\#Origen de datos

Para la realización de esta practica se propone la utilización del
dataset Cervical Cancer Risk Classification, un dataset que recoge
diferentes marcadores relacionados con la aparición del cancer de cuello
uterino. El dataset puede encontrarse tanto en el enlace de Kaggle
\url{https://www.kaggle.com/loveall/cervical-cancer-risk-classification}
como en el repositorio UCI
(\url{https://archive.ics.uci.edu/ml/datasets/Cervical+cancer+\%28Risk+Factors\%29})
Se trata de un dataset de origen público publicado por el Hospital
Universitario de Caracas, Venezuela.

\#\#Motivación

Pese a que se trata de unos de los canceres más prevenibles y el número
de nuevos casos ha disminuido constantenmente en los últimos años,
aproximadamente 4000 muejeres en Estado Unidos y 300000 en todo el mundo
son diagnosticadas cada año. La mortalidad provocada por este cancer se
ha reducido de manera notable gracias a las pruebas de cribado. No
obstante, numerosos estudios confirman que el nivel de probreza y otros
factores socioeconómicos están relacionados con bajas tasas de cribado.
Por todo esto consideramos que se trata de un tema de actualidad que
requiere de mucha colaboración ya que entender bien la información de
que disponemos puede facilitar el estudio de futuros trabajos.

\#\#Carga incial

Una vez obtenido el dataset el primer paso sería cargar los datos,
entender de que tipo de información disponemos, analizar las variables y
realizar un primer análisis de la información para ver la calidad de los
datos y obtener las primeras conclusiones.

Para poder ilustrar el ejemplo, cargamos el fichero de datos

\begin{Shaded}
\begin{Highlighting}[]
\NormalTok{cervical\_data }\OtherTok{\textless{}{-}} \FunctionTok{read.csv}\NormalTok{(}\StringTok{\textquotesingle{}kag\_risk\_factors\_cervical\_cancer.csv\textquotesingle{}}\NormalTok{,}\AttributeTok{stringsAsFactors =} \ConstantTok{FALSE}\NormalTok{)}
\NormalTok{rows}\OtherTok{=}\FunctionTok{dim}\NormalTok{(cervical\_data)[}\DecValTok{1}\NormalTok{]}
\end{Highlighting}
\end{Shaded}

Instalamos y cargamos las librerías ggplot2 y dplry

\begin{Shaded}
\begin{Highlighting}[]
\CommentTok{\# https://cran.r{-}project.org/web/packages/ggplot2/index.html}
\ControlFlowTok{if}\NormalTok{ (}\SpecialCharTok{!}\FunctionTok{require}\NormalTok{(}\StringTok{\textquotesingle{}ggplot2\textquotesingle{}}\NormalTok{)) }\FunctionTok{install.packages}\NormalTok{(}\StringTok{\textquotesingle{}ggplot2\textquotesingle{}}\NormalTok{); }\FunctionTok{library}\NormalTok{(}\StringTok{\textquotesingle{}ggplot2\textquotesingle{}}\NormalTok{)}
\CommentTok{\# https://cran.r{-}project.org/web/packages/dplyr/index.html}
\ControlFlowTok{if}\NormalTok{ (}\SpecialCharTok{!}\FunctionTok{require}\NormalTok{(}\StringTok{\textquotesingle{}dplyr\textquotesingle{}}\NormalTok{)) }\FunctionTok{install.packages}\NormalTok{(}\StringTok{\textquotesingle{}dplyr\textquotesingle{}}\NormalTok{); }\FunctionTok{library}\NormalTok{(}\StringTok{\textquotesingle{}dplyr\textquotesingle{}}\NormalTok{)}
\end{Highlighting}
\end{Shaded}

\begin{center}\rule{0.5\linewidth}{0.5pt}\end{center}

\hypertarget{integraciuxf3n-y-selecciuxf3n-de-los-datos-de-interuxe9s}{%
\section{2.Integración y selección de los datos de
interés}\label{integraciuxf3n-y-selecciuxf3n-de-los-datos-de-interuxe9s}}

\begin{center}\rule{0.5\linewidth}{0.5pt}\end{center}

\#\#Análisis de variables

En primer lugar comprobamos la estructura del fichero:

\begin{Shaded}
\begin{Highlighting}[]
\FunctionTok{str}\NormalTok{(cervical\_data)}
\end{Highlighting}
\end{Shaded}

\begin{verbatim}
## 'data.frame':    858 obs. of  36 variables:
##  $ Age                               : int  18 15 34 52 46 42 51 26 45 44 ...
##  $ Number.of.sexual.partners         : chr  "4.0" "1.0" "1.0" "5.0" ...
##  $ First.sexual.intercourse          : chr  "15.0" "14.0" "?" "16.0" ...
##  $ Num.of.pregnancies                : chr  "1.0" "1.0" "1.0" "4.0" ...
##  $ Smokes                            : chr  "0.0" "0.0" "0.0" "1.0" ...
##  $ Smokes..years.                    : chr  "0.0" "0.0" "0.0" "37.0" ...
##  $ Smokes..packs.year.               : chr  "0.0" "0.0" "0.0" "37.0" ...
##  $ Hormonal.Contraceptives           : chr  "0.0" "0.0" "0.0" "1.0" ...
##  $ Hormonal.Contraceptives..years.   : chr  "0.0" "0.0" "0.0" "3.0" ...
##  $ IUD                               : chr  "0.0" "0.0" "0.0" "0.0" ...
##  $ IUD..years.                       : chr  "0.0" "0.0" "0.0" "0.0" ...
##  $ STDs                              : chr  "0.0" "0.0" "0.0" "0.0" ...
##  $ STDs..number.                     : chr  "0.0" "0.0" "0.0" "0.0" ...
##  $ STDs.condylomatosis               : chr  "0.0" "0.0" "0.0" "0.0" ...
##  $ STDs.cervical.condylomatosis      : chr  "0.0" "0.0" "0.0" "0.0" ...
##  $ STDs.vaginal.condylomatosis       : chr  "0.0" "0.0" "0.0" "0.0" ...
##  $ STDs.vulvo.perineal.condylomatosis: chr  "0.0" "0.0" "0.0" "0.0" ...
##  $ STDs.syphilis                     : chr  "0.0" "0.0" "0.0" "0.0" ...
##  $ STDs.pelvic.inflammatory.disease  : chr  "0.0" "0.0" "0.0" "0.0" ...
##  $ STDs.genital.herpes               : chr  "0.0" "0.0" "0.0" "0.0" ...
##  $ STDs.molluscum.contagiosum        : chr  "0.0" "0.0" "0.0" "0.0" ...
##  $ STDs.AIDS                         : chr  "0.0" "0.0" "0.0" "0.0" ...
##  $ STDs.HIV                          : chr  "0.0" "0.0" "0.0" "0.0" ...
##  $ STDs.Hepatitis.B                  : chr  "0.0" "0.0" "0.0" "0.0" ...
##  $ STDs.HPV                          : chr  "0.0" "0.0" "0.0" "0.0" ...
##  $ STDs..Number.of.diagnosis         : int  0 0 0 0 0 0 0 0 0 0 ...
##  $ STDs..Time.since.first.diagnosis  : chr  "?" "?" "?" "?" ...
##  $ STDs..Time.since.last.diagnosis   : chr  "?" "?" "?" "?" ...
##  $ Dx.Cancer                         : int  0 0 0 1 0 0 0 0 1 0 ...
##  $ Dx.CIN                            : int  0 0 0 0 0 0 0 0 0 0 ...
##  $ Dx.HPV                            : int  0 0 0 1 0 0 0 0 1 0 ...
##  $ Dx                                : int  0 0 0 0 0 0 0 0 1 0 ...
##  $ Hinselmann                        : int  0 0 0 0 0 0 1 0 0 0 ...
##  $ Schiller                          : int  0 0 0 0 0 0 1 0 0 0 ...
##  $ Citology                          : int  0 0 0 0 0 0 0 0 0 0 ...
##  $ Biopsy                            : int  0 0 0 0 0 0 1 0 0 0 ...
\end{verbatim}

Se puede observar como existen 858 observación y un total de 36
variables que corresponden a los diferentes indicadores a tener en
cuenta en la identificación del cancer de cuello de útero. La
descripción de las variables es la siguiente:

\textbf{Age} Edad del paciente

\textbf{Number of sexual partners} Numero de personas con las que ha
mantenido relaciones

\textbf{First sexual intercourse} Edad a la que mantuvo su primera
relación

\textbf{Num of pregnancies} Número de embarazos

\textbf{Smokes} Indicador de si es fumador (1/0)

\textbf{Smokes years} Número de años siendo fumadora

\textbf{Smokes pack years} Valor que cuantifica el consumo de tabaco

\textbf{Hormonal contraceptives} Utilización de anticonceptivos
hormonales (1/0)

\textbf{Hormonal Contraceptives years} Utilización de anticonceptivos
hormonales en años

\textbf{IUD} Dispositivo intrauterino (1/0)

\textbf{IUD years} Dispositivo intrauterino en años

\textbf{STDs} Enfermedades de transmisión sexual (ETS)

\textbf{STDs number} Número de enfermedades (ETS)

\textbf{STDs condylomatosis} ETS condilomatosis (1/0)

\textbf{STDs cervical.condylomatosis} ETS condilomatosis cervical (1/0)

\textbf{STDs vaginal.condylomatosis} ETS condilomatosis vaginal (1/0)

\textbf{STDs vulvo.perineal.condylomatosis} ETS condilomatosis vulvo
perineal (1/0)

\textbf{STDs syphilis} ETS sifilis (1/0)

\textbf{STDs pelvic.inflammatory.disease} ETS enfermedad pelvica
inflamatoria (1/0)

\textbf{STDs genital.herpes} ETS herpes (1/0)

\textbf{STDs molluscum.contagiosum} ETS molusco contagioso (1/0)

\textbf{STDs AIDS} ETS SIDA (1/0)

\textbf{STDs HIV} ETS VIH (1/0)

\textbf{STDs Hepatitis.B} ETS Hepatitis B (1/0)

\textbf{STDs HPV} ETS Virus del papiloma humano (1/0)

\textbf{STDs..Number.of.diagnosis} Número de ETS diagnosticadas

\textbf{STDs..Time.since.first.diagnosis} Tiempo desde la primera vez
diagnosticada

\textbf{STDs..Time.since.last.diagnosis} Tiempo desde la última vez
diagnosticada

\textbf{Dx.Cancer} Diagnostico Cancer (1/0)

\textbf{Dx.CIN} Diagnostico lesiones precancerosas(1/0)

\textbf{Dx.HPV} Diagnostico VPH (Virus del papiloma humano)

\textbf{Dx} Existe diagnostico

\textbf{Hinselmann} Prueba de Hinselmann (1/0)

\textbf{Schiller} Prueba de Schiller(1/0)

\textbf{Citology} Prueba Citologia (1/0)

\textbf{Biopsy} Prueba Biopsia (1/0)

\hypertarget{selecciuxf3n-de-variables}{%
\subsubsection{Selección de variables}\label{selecciuxf3n-de-variables}}

Una vez identificadas las variables, el siguiente paso sería realizar un
proceso de limpieza de los datos. Descartarlos datos con mala calidad o
que, a priori, pensemos no aportan mucho valor en los resultados
esperados. En este caso concreto podríamos eliminar las columnas
relacionadas a los diferentes tipos de ETS dejando únicamente el
indicador general que informa si ha padecido alguna enfermedad de este
tipo y guardar los datos en un subset adicional por si fuera necesario
lanzar análisis más detallados en el futuro.

\begin{Shaded}
\begin{Highlighting}[]
\NormalTok{cervical\_data\_subset}\OtherTok{\textless{}{-}}\NormalTok{cervical\_data[}\FunctionTok{c}\NormalTok{(}\DecValTok{1}\SpecialCharTok{:}\DecValTok{13}\NormalTok{,}\DecValTok{26}\SpecialCharTok{:}\DecValTok{36}\NormalTok{)]}
\end{Highlighting}
\end{Shaded}

\begin{center}\rule{0.5\linewidth}{0.5pt}\end{center}

\hypertarget{limpieza-de-datos}{%
\section{3.Limpieza de datos}\label{limpieza-de-datos}}

\begin{center}\rule{0.5\linewidth}{0.5pt}\end{center}

\#\#Transformación de variables

Para poder obtener resultados estadísticos de los datos es necesario
transformar algunos datos que viene como string a tipo entero.

\begin{Shaded}
\begin{Highlighting}[]
\NormalTok{cervical\_data\_subset}\SpecialCharTok{$}\NormalTok{Number.of.sexual.partners}\OtherTok{\textless{}{-}}\FunctionTok{as.integer}\NormalTok{(cervical\_data\_subset}\SpecialCharTok{$}\NormalTok{Number.of.sexual.partners)}
\NormalTok{cervical\_data\_subset}\SpecialCharTok{$}\NormalTok{First.sexual.intercourse}\OtherTok{\textless{}{-}}\FunctionTok{as.integer}\NormalTok{(cervical\_data\_subset}\SpecialCharTok{$}\NormalTok{First.sexual.intercourse)}
\NormalTok{cervical\_data\_subset}\SpecialCharTok{$}\NormalTok{Smokes}\OtherTok{\textless{}{-}}\FunctionTok{as.integer}\NormalTok{(cervical\_data\_subset}\SpecialCharTok{$}\NormalTok{Smokes)}
\NormalTok{cervical\_data\_subset}\SpecialCharTok{$}\NormalTok{Smokes..years.}\OtherTok{\textless{}{-}}\FunctionTok{as.integer}\NormalTok{(cervical\_data\_subset}\SpecialCharTok{$}\NormalTok{Smokes..years.)}
\NormalTok{cervical\_data\_subset}\SpecialCharTok{$}\NormalTok{Num.of.pregnancies}\OtherTok{\textless{}{-}}\FunctionTok{as.integer}\NormalTok{(cervical\_data\_subset}\SpecialCharTok{$}\NormalTok{Num.of.pregnancies)}
\end{Highlighting}
\end{Shaded}

Otro paso dentro de la limpieza de datos podría ser eliminar los datos
redundantes o duplicados, en este caso concreto, al no tener ningún
cambo que identifique al paciente como único aceptaremos la premisa de
que diferentes pacientes pueden peteir diferentes valores y por lo tanto
no eliminaremos los registros duplicados.

\#\#Eliminación de valores nulos o no informados

Siguiendo con el proceso de limpieza, el siguiente paso sería
identificar los valores nulos, no informados, o informados con
caracteres extraños.

Estadísticas de valores vacíos

\begin{Shaded}
\begin{Highlighting}[]
\FunctionTok{colSums}\NormalTok{(}\FunctionTok{is.na}\NormalTok{(cervical\_data\_subset))}
\end{Highlighting}
\end{Shaded}

\begin{verbatim}
##                              Age        Number.of.sexual.partners 
##                                0                               26 
##         First.sexual.intercourse               Num.of.pregnancies 
##                                7                               56 
##                           Smokes                   Smokes..years. 
##                               13                               13 
##              Smokes..packs.year.          Hormonal.Contraceptives 
##                                0                                0 
##  Hormonal.Contraceptives..years.                              IUD 
##                                0                                0 
##                      IUD..years.                             STDs 
##                                0                                0 
##                    STDs..number.        STDs..Number.of.diagnosis 
##                                0                                0 
## STDs..Time.since.first.diagnosis  STDs..Time.since.last.diagnosis 
##                                0                                0 
##                        Dx.Cancer                           Dx.CIN 
##                                0                                0 
##                           Dx.HPV                               Dx 
##                                0                                0 
##                       Hinselmann                         Schiller 
##                                0                                0 
##                         Citology                           Biopsy 
##                                0                                0
\end{verbatim}

Estadísticas de valores nulos

\begin{Shaded}
\begin{Highlighting}[]
\FunctionTok{colSums}\NormalTok{(cervical\_data\_subset}\SpecialCharTok{==}\StringTok{""}\NormalTok{)}
\end{Highlighting}
\end{Shaded}

\begin{verbatim}
##                              Age        Number.of.sexual.partners 
##                                0                               NA 
##         First.sexual.intercourse               Num.of.pregnancies 
##                               NA                               NA 
##                           Smokes                   Smokes..years. 
##                               NA                               NA 
##              Smokes..packs.year.          Hormonal.Contraceptives 
##                                0                                0 
##  Hormonal.Contraceptives..years.                              IUD 
##                                0                                0 
##                      IUD..years.                             STDs 
##                                0                                0 
##                    STDs..number.        STDs..Number.of.diagnosis 
##                                0                                0 
## STDs..Time.since.first.diagnosis  STDs..Time.since.last.diagnosis 
##                                0                                0 
##                        Dx.Cancer                           Dx.CIN 
##                                0                                0 
##                           Dx.HPV                               Dx 
##                                0                                0 
##                       Hinselmann                         Schiller 
##                                0                                0 
##                         Citology                           Biopsy 
##                                0                                0
\end{verbatim}

Estadísticas de valores informados con `?'

\begin{Shaded}
\begin{Highlighting}[]
\FunctionTok{colSums}\NormalTok{(cervical\_data\_subset}\SpecialCharTok{==}\StringTok{"?"}\NormalTok{)}
\end{Highlighting}
\end{Shaded}

\begin{verbatim}
##                              Age        Number.of.sexual.partners 
##                                0                               NA 
##         First.sexual.intercourse               Num.of.pregnancies 
##                               NA                               NA 
##                           Smokes                   Smokes..years. 
##                               NA                               NA 
##              Smokes..packs.year.          Hormonal.Contraceptives 
##                               13                              108 
##  Hormonal.Contraceptives..years.                              IUD 
##                              108                              117 
##                      IUD..years.                             STDs 
##                              117                              105 
##                    STDs..number.        STDs..Number.of.diagnosis 
##                              105                                0 
## STDs..Time.since.first.diagnosis  STDs..Time.since.last.diagnosis 
##                              787                              787 
##                        Dx.Cancer                           Dx.CIN 
##                                0                                0 
##                           Dx.HPV                               Dx 
##                                0                                0 
##                       Hinselmann                         Schiller 
##                                0                                0 
##                         Citology                           Biopsy 
##                                0                                0
\end{verbatim}

Una vez identificados, se procede a aplicar el tratamiento de falta de
datos que más se adapte a cada tipo de dato teniendo en cuenta que un
alto número de valores sin informar puede ser un indicador de
incosistencia.Con ciertas variables podemos tomar el valor medio de los
datos o el valor más repetido, en otros casos podemos tomar valores
condicionales en base a la edad, por ejemplo si es fumador o no, fijando
el valor NO para menores de 18 años. En otros casos es posible que nos
convenga mantener los valores nulos y trabajar únicamente con el subset
de datos informados como puede ser el indicador de si una persona estaba
embarazada o no.

Antes de entrar a realizar tareas de transformación es conveniente
categorizar las variables y comprobar que están informadas de un modo
coherente. Por ejemplo, comprobar que los campos de edad no tiene
valores muy elevados o caracteres en lugar de números, eliminar los
espacios de las variables de tipo string, comprobar que los indicadores
se informar únicamente con valores de 0 y 1\ldots{}

Transformación de las variables Number of sexual partners y first sexual
intercourse por la media:

\begin{Shaded}
\begin{Highlighting}[]
\NormalTok{cervical\_data\_subset}\SpecialCharTok{$}\NormalTok{Number.of.sexual.partners[}\FunctionTok{is.na}\NormalTok{(cervical\_data\_subset}\SpecialCharTok{$}\NormalTok{Number.of.sexual.partners)] }\OtherTok{\textless{}{-}}\FunctionTok{mean}\NormalTok{(cervical\_data\_subset}\SpecialCharTok{$}\NormalTok{Number.of.sexual.partners,}\AttributeTok{na.rm=}\NormalTok{T)}
\NormalTok{cervical\_data\_subset}\SpecialCharTok{$}\NormalTok{First.sexual.intercourse[}\FunctionTok{is.na}\NormalTok{(cervical\_data\_subset}\SpecialCharTok{$}\NormalTok{First.sexual.intercourse)] }\OtherTok{\textless{}{-}}\FunctionTok{mean}\NormalTok{(cervical\_data\_subset}\SpecialCharTok{$}\NormalTok{First.sexual.intercourse,}\AttributeTok{na.rm=}\NormalTok{T)}
\end{Highlighting}
\end{Shaded}

Filtramos todos los valores con el registro IUD =`?' y les aplicamos el
valor NA, se trata de una variable muy importante en nuestro estudio y
no es fácilmente interpolable por lo que al ser un número elevado de
muestras puede afectar al análisis de otras variables.

\begin{Shaded}
\begin{Highlighting}[]
\NormalTok{cervical\_data\_subset}\SpecialCharTok{$}\NormalTok{IUD[cervical\_data\_subset}\SpecialCharTok{$}\NormalTok{IUD}\SpecialCharTok{==}\StringTok{"?"}\NormalTok{] }\OtherTok{\textless{}{-}} \StringTok{"NA"}
\end{Highlighting}
\end{Shaded}

\#\#Identificación y tratamiento de valores extremos

A continuación se procede a normalizar el campo edad para poder realizar
representaciónes gráficas de un modo más sencillo

\begin{Shaded}
\begin{Highlighting}[]
\FunctionTok{summary}\NormalTok{(cervical\_data\_subset[,}\StringTok{"Age"}\NormalTok{])}
\end{Highlighting}
\end{Shaded}

\begin{verbatim}
##    Min. 1st Qu.  Median    Mean 3rd Qu.    Max. 
##   13.00   20.00   25.00   26.82   32.00   84.00
\end{verbatim}

\begin{Shaded}
\begin{Highlighting}[]
\NormalTok{cervical\_data\_subset[}\StringTok{"Age\_segment"}\NormalTok{] }\OtherTok{\textless{}{-}} \FunctionTok{cut}\NormalTok{(cervical\_data\_subset}\SpecialCharTok{$}\NormalTok{Age, }\AttributeTok{breaks =} \FunctionTok{c}\NormalTok{(}\DecValTok{0}\NormalTok{,}\DecValTok{10}\NormalTok{,}\DecValTok{20}\NormalTok{,}\DecValTok{30}\NormalTok{,}\DecValTok{40}\NormalTok{,}\DecValTok{50}\NormalTok{,}\DecValTok{60}\NormalTok{,}\DecValTok{70}\NormalTok{,}\DecValTok{100}\NormalTok{), }\AttributeTok{labels =} \FunctionTok{c}\NormalTok{(}\StringTok{"13{-}21"}\NormalTok{, }\StringTok{"22{-}30"}\NormalTok{, }\StringTok{"31{-}39"}\NormalTok{, }\StringTok{"40{-}48"}\NormalTok{,}\StringTok{"49{-}57"}\NormalTok{,}\StringTok{"58{-}66"}\NormalTok{,}\StringTok{"67{-}75"}\NormalTok{,}\StringTok{"76{-}84"}\NormalTok{))}
\end{Highlighting}
\end{Shaded}

\begin{center}\rule{0.5\linewidth}{0.5pt}\end{center}

\hypertarget{anuxe1lisis-de-los-datos}{%
\section{4. Análisis de los datos}\label{anuxe1lisis-de-los-datos}}

\begin{center}\rule{0.5\linewidth}{0.5pt}\end{center}

\#\#Selección de los datos

Precisamos un primer set de datos para continuar con los análisis
escogemos las variables más significativas como edad segmentada,
variables relacionadas con el tabaco, los metodos hormonales y
dispositivos intrauterinos Y

\begin{Shaded}
\begin{Highlighting}[]
\NormalTok{analisis\_subset}\OtherTok{\textless{}{-}}\NormalTok{cervical\_data\_subset[}\FunctionTok{c}\NormalTok{(}\DecValTok{25}\NormalTok{,}\DecValTok{5}\SpecialCharTok{:}\DecValTok{13}\NormalTok{)]}
\FunctionTok{str}\NormalTok{(analisis\_subset)}
\end{Highlighting}
\end{Shaded}

\begin{verbatim}
## 'data.frame':    858 obs. of  10 variables:
##  $ Age_segment                    : Factor w/ 8 levels "13-21","22-30",..: 2 2 4 6 5 5 6 3 5 5 ...
##  $ Smokes                         : int  0 0 0 1 0 0 1 0 0 1 ...
##  $ Smokes..years.                 : int  0 0 0 37 0 0 34 0 0 1 ...
##  $ Smokes..packs.year.            : chr  "0.0" "0.0" "0.0" "37.0" ...
##  $ Hormonal.Contraceptives        : chr  "0.0" "0.0" "0.0" "1.0" ...
##  $ Hormonal.Contraceptives..years.: chr  "0.0" "0.0" "0.0" "3.0" ...
##  $ IUD                            : chr  "0.0" "0.0" "0.0" "0.0" ...
##  $ IUD..years.                    : chr  "0.0" "0.0" "0.0" "0.0" ...
##  $ STDs                           : chr  "0.0" "0.0" "0.0" "0.0" ...
##  $ STDs..number.                  : chr  "0.0" "0.0" "0.0" "0.0" ...
\end{verbatim}

\begin{Shaded}
\begin{Highlighting}[]
\FunctionTok{ggplot}\NormalTok{(}\AttributeTok{data =}\NormalTok{ analisis\_subset, }\FunctionTok{aes}\NormalTok{(}\AttributeTok{x =}\NormalTok{ Age\_segment, }\AttributeTok{y =}\NormalTok{ Smokes..years., }\AttributeTok{colour =}\NormalTok{ Age\_segment)) }\SpecialCharTok{+}
  \FunctionTok{geom\_boxplot}\NormalTok{() }\SpecialCharTok{+}
  \FunctionTok{geom\_point}\NormalTok{() }\SpecialCharTok{+}
  \FunctionTok{theme\_bw}\NormalTok{() }\SpecialCharTok{+}
  \FunctionTok{theme}\NormalTok{(}\AttributeTok{legend.position =} \StringTok{"none"}\NormalTok{)}
\end{Highlighting}
\end{Shaded}

\includegraphics{Practica-2---Limpieza-de-datos_files/figure-latex/unnamed-chunk-13-1.pdf}

\begin{center}\rule{0.5\linewidth}{0.5pt}\end{center}

\hypertarget{representaciuxf3n-de-los-resultados}{%
\section{5. Representación de los
resultados}\label{representaciuxf3n-de-los-resultados}}

\begin{center}\rule{0.5\linewidth}{0.5pt}\end{center}

\#\#Representación gráfica Para entender los datos más en profundidad
procedemos a realizar varias representaciones gráficas para obtener una
visión más general de como se distribuye la información, de este modo
nos será más sencillo identificar la naturaleza de los datos.

Distribución de los grupos de edad

\begin{Shaded}
\begin{Highlighting}[]
 \FunctionTok{plot}\NormalTok{(cervical\_data\_subset}\SpecialCharTok{$}\NormalTok{Age\_segment)}
\end{Highlighting}
\end{Shaded}

\includegraphics{Practica-2---Limpieza-de-datos_files/figure-latex/unnamed-chunk-14-1.pdf}
Representación de los grupos de edad en base a si son fumadores o no:

\begin{Shaded}
\begin{Highlighting}[]
\FunctionTok{ggplot}\NormalTok{(}\AttributeTok{data=}\NormalTok{cervical\_data\_subset[}\DecValTok{1}\SpecialCharTok{:}\NormalTok{rows,],}\FunctionTok{aes}\NormalTok{(}\AttributeTok{x=}\NormalTok{Smokes ,}\AttributeTok{fill=}\NormalTok{Age\_segment))}\SpecialCharTok{+}\FunctionTok{geom\_bar}\NormalTok{()}
\end{Highlighting}
\end{Shaded}

\includegraphics{Practica-2---Limpieza-de-datos_files/figure-latex/unnamed-chunk-15-1.pdf}
En la grafica vemos como muchos más casos corresponden a personas no
fumadoras independiente del rango de edad ya que este se distribuye de
acuerdo a la distribución anterior por grupos de edad, en todos los
grupos encontramos personas fumadoras excepto en el grupo 0 de 13 a 21
años. En este caso quizás podría darnos pie a pensar que el tabaco no es
un factor de riesgo en este tipo de cancer pero habría que confirmarlo
con análisis más avanzados.

Representación de los grupos de edad en base a si son fumadores o no:

\begin{Shaded}
\begin{Highlighting}[]
\FunctionTok{ggplot}\NormalTok{(}\AttributeTok{data=}\NormalTok{cervical\_data\_subset[}\DecValTok{1}\SpecialCharTok{:}\NormalTok{rows,],}\FunctionTok{aes}\NormalTok{(}\AttributeTok{x=}\NormalTok{IUD ,}\AttributeTok{fill=}\NormalTok{Age\_segment))}\SpecialCharTok{+}\FunctionTok{geom\_bar}\NormalTok{()}
\end{Highlighting}
\end{Shaded}

\includegraphics{Practica-2---Limpieza-de-datos_files/figure-latex/unnamed-chunk-16-1.pdf}

\begin{center}\rule{0.5\linewidth}{0.5pt}\end{center}

\hypertarget{resoluciuxf3n-de-problemas}{%
\section{6. Resolución de problemas}\label{resoluciuxf3n-de-problemas}}

\begin{center}\rule{0.5\linewidth}{0.5pt}\end{center}

Una vez analizado el dataset en profundidad se abre un amplio número de
posibilidades.Por ejemplo, seríamos capaces de identificar diferentes
agrupaciones en base a las variables descritas y lanzar las campañas de
concienciación personalizadas a los grupos más numerosos. Una vez
establecida la clasificación sería necesario \textbf{analizar la calidad
del modelo} para identificar que efectivamente los sujetos dentro de
cada grupo tienen similitudes, existen diferencias entre los diferentes
grupos y la distancia de cada muestra con el centro de su cluster es
apropiada para definir un buen nivel de calidad. Por último, una vez
establecido el modelo y teniendo en cuenta que todas las mujeres que se
encuentran en el dataset han sufrido cáncer. Si hemos sido capaces de
identificar grupos con características comunes con un aceptable nivel de
calidad, seremos capaces de \textbf{lanzar campañas de prevención
personalizadas} lo más efectivas posible. Un posible resultado podría
ser por un lado, proponer a las mujeres de entre 30 y 40 que no utilicen
dispositivo intrauterino; por otro lado, proponer la realización de
pruebas diagnosticas gratuitas a las mujeres de entre 20 y 30 que nunca
se han realizado pruebas con anterioridad y una tercera campaña podría
ir dirigida a las personas mayores de 40 años que han fumado durante más
de 5 años para que dejen de fumar y concienciarles de los riesgos de
seguir fumando.

\end{document}
